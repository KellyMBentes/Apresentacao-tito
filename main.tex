\documentclass[aspectratio=169]{beamer}	 	

\usetheme{Pittsburgh}
\usecolortheme{default}
\usefonttheme[onlymath]{serif}			% para fontes matemáticas
% Enconte mais temas e cores em http://www.hartwork.org/beamer-theme-matrix/ 
% Veja também http://deic.uab.es/~iblanes/beamer_gallery/index.html
\usepackage{media9}
% Customizações de Cores: fg significa cor do texto e bg é cor do fundo
\setbeamercolor{normal text}{fg=black}
\setbeamercolor{alerted text}{fg=red}
\setbeamercolor{author}{fg=blue}
\setbeamercolor{institute}{fg=blue}
\setbeamercolor{date}{fg=green}
\setbeamercolor{frametitle}{fg=red}
\setbeamercolor{framesubtitle}{fg=brown}
\setbeamercolor{block title}{bg=blue, fg=white}		%Cor do título
\setbeamercolor{block body}{bg=gray, fg=darkgray}	%Cor do texto (bg= fundo; fg=texto)

% ---
% PACOTES
% ---
\usepackage[alf]{abntex2cite}		% Citações padrão ABNT
\usepackage[brazil]{babel}		% Idioma do documento
\usepackage{color}			% Controle das cores
\usepackage[T1]{fontenc}		% Selecao de codigos de fonte.
\usepackage{graphicx}			% Inclusão de gráficos
\usepackage[utf8]{inputenc}		% Codificacao do documento (conversão automática dos acentos)
\usepackage{txfonts}			% Fontes virtuais
% ---

% --- Informações do documento ---
\title{Acessibilidade e Inclusão}
\author{Lucas Tito}
\institute{Grupo de Apoio a Pesquisas Em Acessibilidade e Inclusão - GAPAI}
\date{\today, v-1.9.6}
% ---

% ----------------- INÍCIO DO DOCUMENTO --------------------------------------
\begin{document}

% ----------------- NOVO SLIDE --------------------------------
\begin{frame}

\titlepage

\end{frame}

% ----------------- NOVO SLIDE --------------------------------
\begin{frame}{Sumário}
\tableofcontents
\end{frame}

% ----------------- NOVO SLIDE --------------------------------

\begin{frame}{Qual nome usar?}

Portador de necessidades especiais
pessoa com deficiência x (x = visual, motora, física, ...)
Cego, cadeirante, surdo, ...

\end{frame}

\begin{frame}{Acessibilidade}
Algo é dito acessível, quando toda e qualquer pessoa de forma indistinta possui acesso.

Exemplo:
um livro é dito acessível quando pode ser lido em braille ou se digital, os leitores de tela conseguem reproduzir o conteudo por meio do sintetizador sem quaisquer problemas. Quando as cores usadas foram pensadas para não atrapalhar pessoas daltônicas e de baixa visão. Quando as letras são legíveis e etc.
Um prédio é dito acessível quando possui escadas largas com degraus nivelados, elevadores com portas largas, botões em braille e que indique o andar por feedback sonoro, que tenham rampas, que tenha piso tátil e etc.

\end{frame}

\begin{frame}{Inclusão}
Algo é dito inclusivo quando todas as pessoas tem igualdade na participação dos movimentos, contextos, ambientes e etc, sem distinções e sem segregação.

Exemplos:
Uma aula é dita inclusiva quando os professores usam materiais acessíveis, não segregam os alunos, quando as aulas não são oferecidas em turmas separadas para pessoas com deficiência e outra turma para pessoas sem deficiência e etc.
Uma palestra é dita inclusiva, quando o palestrante ao inícil da mesma se pronuncia sem microfone dando assim a oportunidade de pessoas cegas o localizarem no ambiente, quando o palestrante usa imagens e as descreve para o público, quando o contraste das informações no slide favorecem o entendimento de pessoas daltônicas e com baixa visão, quando o local de acesso é acessível, quando ele o palestrante se preocupa em por textos de forma a facilitar que pessoas surdas acompanhem mesmo que o palestrante não saiba libras.
\end{frame}

\begin{frame}{Empatia}
\includemedia[
  width=0.4\linewidth,
  totalheight=0.225\linewidth,
  activate=pageopen,
  passcontext,  %show VPlayer's right-click menu
  addresource=empatia.mp4,
  flashvars={
    %important: same path as in `addresource'
    source=empatia.mp4
  }
]{\fbox{Click!}}{VPlayer.swf}

\end{frame}

\section{Referências}

% --- O comando \allowframebreaks ---
% Se o conteúdo não se encaixa em um quadro, a opção allowframebreaks instrui 
% beamer para quebrá-lo automaticamente entre dois ou mais quadros,
% mantendo o frametitle do primeiro quadro (dado como argumento) e acrescentando 
% um número romano ou algo parecido na continuação.

\begin{frame}[allowframebreaks]{Referências}
\bibliography{abntex2-modelo-references}
\end{frame}

% ----------------- FIM DO DOCUMENTO -----------------------------------------
\end{document}
